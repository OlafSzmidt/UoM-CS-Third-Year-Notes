\documentclass[11pt]{article}
\usepackage[utf8]{inputenc}
\usepackage{fancyhdr}
\usepackage[margin=1.45cm]{geometry}

\title{COMP38120 - Documents, Services and Data on the Web}
\author{Sam Littlefair}

\usepackage{natbib}
\usepackage{graphicx}
\usepackage{amsmath}
\usepackage{textcomp}

\begin{document}

\maketitle


\section{Course Unit Structure}
\begin{itemize}
  \item 20 credit module with 1 exam in Summer worth 60\%.
  \item 2 labs due in Semester 2:
  \begin{enumerate}
     \item Week 3 - 25\%.
     \item Week 11 - 15\%.
    \end{enumerate}
\end{itemize}


\section{Documents on the Web}
The “Traditional” Web: Web	pages,	images,	videos,	audio	files,	etc.
Collectively	known	as	documents - objects	that	can
be	indexed	(so	that	it	can	be	found) and searched for.

\subsection{Information retrieval}
Information retrieval is finding material of an unstructured nature that
satisfies an information need from within large collections (Manning et al 2008).
IR is not just search engines, it can be other things such as voice controlled assistants like Siri, or music recognition like Shazam.
There is no obvious structure, unlike a rigidly structured database - but doesn't mean there's no structure at all, there is document structure (paragraphs, headings..) and mark-up (XML, HTML).

\subsection{Informational need} %REFACTOR THIS SECTION
Informational need is the topic about which the user desires to know more about.

\begin{itemize}
    \item \textbf{Informational:} Need to learn about something (80\% of queries)
    \item \textbf{Navigational:} Need to go to some page (10\%)
    \item \textbf{Transactional:} Need to do something using the Web (10\%)
\end{itemize}
How do we distinguish between them?\\
\newline
\textbf{Navigational}
\begin{itemize}
    \item Company/business/org name
    \item Domain suffix
    \item Length of query (\textless3)
\end{itemize}
\newpage
\textbf{Transactional}
\begin{itemize}
    \item Queries with ‘interact’ terms (buy, chat,…) or ‘obtaining’ terms
(e.g. ‘download’, ‘get’)
    \item Terms related to movies, songs, lyrics, recipes, images, humour
    \begin{itemize}
        \item Image, audio or video collections (‘audio’, ‘images’, ‘video’)
        \item Movies, songs, images, multimedia, compression file extensions
(jpg, zip, …)
        \item Entertainment terms (pictures, games, …)
    \end{itemize}
\end{itemize}
\textbf{Informational}
\begin{itemize}
    \item Use of question words, many phrases
    \item Informational items (‘list of…’)
    \item Query length \textgreater2
    \item Neither navigational nor transactional i.e., easier to identify
N and T (?)
\end{itemize}
\subsection{User Behaviour}
\begin{itemize}
    \item About 3 searches per day for an average user.
    \item 16\% to 20\% of queries that get asked every day have
    never been asked before.
    \item 78\% of queries not modified.
    \item User queries are very short. 38\% of queries single word, 22\% 2 words etc. All full of spelling mistakes!
\end{itemize}

\subsection{Information Retrieval System}
An IR system finds the similarities of a query representation to
document representation. Documents must be indexed – each document is reduced to set of index terms.
\begin{itemize}
    \item Index language: A means of representing documents and queries.
    \item Search is focused on index: Map queries to index i.e. determine similarity of a query to a document via index items.
\end{itemize}
\subsubsection{Indexing process (offline)}
\begin{enumerate}
    \item Acquire documents through crawling, feeds, etc.
    \item Generate index for each document
    \item Store the index in an easy-to-search form
\end{enumerate}
\subsubsection{The retrieval process (online)}
\begin{enumerate}
    \item Assist the user in formulating the query (did you mean, autocomplete..)
    \item Transform the query (“index” the query)
    \item Map the query’s representation against the index
    \item Retrieve and rank the results
    \item Help users browse the results and re-formulate the query
    \item Log user’s actions etc.
\end{enumerate}

\subsection{Representation}
\subsubsection{PubMed - Medical Publications}
MeSH (Medical Subject Headings) is the a manually controlled
vocabulary used for indexing articles for PubMed.\\
Very hard to maintain, indexing can take months.

\subsubsection{Bag of Words (BoW)}
An indexing model which reduces the document into a map of words and their frequencies.\\
\newline
\textbf{Pros}
\begin{itemize}
    \item IR matching is about the degree of overlap between a query and document representations so it's efficient.
\end{itemize}
\textbf{Cons}
\begin{itemize}
    \item Meaning could be lost without order, but not a huge deal.
    \item Meaning could be lost without context.
    \item Might not work for all languages.
    \item Negations are lost.
\end{itemize}

\subsubsection{Term-document Matrix}
Store the number of occurrences in a matrix where one dimension is terms, and the other is the documents.
\[
\bordermatrix{ & Doc1 & Doc2 & Doc3 \cr
      complexity & 4 & 0 & 3 \cr
      algorithm & 0 & 0 & 2 \cr
      entropy & 1 & 1 & 0 } \qquad
\]

\subsubsection{Term-document Incidence Matrix}
Just store a 1 if it occurs, 0 otherwise.
\[
\bordermatrix{ & Doc1 & Doc2 & Doc3 \cr
      complexity & 1 & 0 & 1 \cr
      algorithm & 0 & 0 & 1 \cr
      entropy & 1 & 1 & 0 } \qquad
\]
\newline\newline
The problem with matrix approach is the matrices will be massive. With 1 million documents, 1000 words per document, and 500k unique words you'd have a matrix of size 500 billion.

\subsubsection{Inverted Index}
For each term, store a list of all documents that contain it.\\
The document list for each term will be of varying length so we can use linked lists or variable sized arrays.\\
\newline
\newcommand{\sep}{\hspace*{.5em}}
  \noindent
  Brutus \textrightarrow \hspace{8.4mm}$\fbox{1} \sep \fbox{2} \sep \fbox{4} \sep \fbox{11} \sep \fbox{31} \sep \fbox{45} \sep \fbox{173} \sep \fbox{174}$\\ \newline
  Caesar \textrightarrow \hspace{8.4mm}$\fbox{1} \sep \fbox{2} \sep \fbox{4} \sep \fbox{5} \sep \fbox{6} \sep \fbox{16} \sep \fbox{57} \sep \fbox{132}$\\ \newline
  Calpurnia \textrightarrow \hspace{3mm}$\fbox{1} \sep \fbox{2} \sep \fbox{2} \sep \fbox{31} \sep \fbox{54} \sep \fbox{101}$

\newline
\textbf{Indexer steps}\\
\newline
\textbf{1. Token sequence}
\begin{itemize}
    \item Create a sequence of Token, Document ID pairs.
\end{itemize}
\textbf{2. Sort}
\begin{itemize}
    \item Sort by terms, then by document ID.
\end{itemize}
\textbf{3. Dictionary \& Postings}
\begin{itemize}
    \item Multiple term entries in a single document are merged
    \item Split into Dictionary and Postings
    \item Document frequency (for each term) is also added
\end{itemize}
\end{document}