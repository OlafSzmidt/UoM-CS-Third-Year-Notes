\documentclass[11pt]{article}
\usepackage[utf8]{inputenc}
\usepackage{fancyhdr}
\usepackage[margin=1.25cm]{geometry}

\title{COMP37111 - Advanced Computer Graphics Notes}
\author{Sam Littlefair}

\usepackage{natbib}
\usepackage{graphicx}
\usepackage{hyperref}
\usepackage{enumerate}

\begin{document}

\maketitle


\section{Course Unit Structure}
\begin{itemize}
  \item 10 credit module. Online exam 75\%.

  \item 30 hour individual lab marked out of 20, due Friday 7 Dec 2018.
  \begin{itemize}
    \item Design  and implement a simulation with particle systems.
    \item Model real-time behaviour of a particle system.
    \item Give analysis of performance, real-time interaction with varying number of particles.
  \end{itemize}
\end{itemize}

\begin{itemize}
\item The course is split into 4 main sections, the first 2 by Toby Howard:
\begin{enumerate}
  \item \textit{Generative Modelling:} Creating 3D models and textures from sets of rules. (Particle Systems, Fractals..)
  \item \textit{Modelling and model acquisition:} Where to get meshes. (Laser scanning, Triangulation..)
\end{enumerate}

\item And the last 2 by Steve Pettifer:
\begin{enumerate}
  \setcounter{enumi}{2}
  \item \textit{Global illumination:} The rendering equation, ray tracing, luminosity..
  \item \textit{Real-time rendering:} Maximising performance (i.e. frame-rate) using methods like culling.
\end{enumerate}
\end{itemize}
\section{Resources}
\begin{itemize}
  \item Real-Time Rendering book. (Tomas Möller, Eric Haines, Naty Hoffman)
  \item \href{http://www.realtimerendering.com/}{Real-Time Rendering website.}
\end{itemize}
\end{document}
